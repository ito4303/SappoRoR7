\documentclass[dvipdfmx,12pt]{beamer}
\usepackage{bxdpx-beamer}
\usepackage{minijs}
%\usepackage{helvet}
%\usepackage[helvet]{sfmath}
\usepackage{cmbright}
%\usepackage{iwona}

\renewcommand{\kanjifamilydefault}{\gtdefault}

\begin{document}

\title{RStanで状態空間モデル}
\author{伊東宏樹}
\date{2016-10-29}
\maketitle

%% 状態空間モデル
\begin{frame}{状態空間モデル}
\begin{itemize}
\item 潜在状態
\item 観測値
\end{itemize}

\end{frame}

%% DLM
\begin{frame}{動的線形モデル}
  観測方程式
  \begin{align*}
    y_t &\sim N(F^{\prime} \theta_t, V)
  \end{align*}

  状態方程式
  \begin{align*}
    \theta_t &\sim N(G \theta_{t-1}, W)
  \end{align*}

  初期値
  \begin{align*}
    \theta_0 &\sim N(m_0, C_0)
  \end{align*}

\end{frame}

%% トレンドモデル
\begin{frame}{トレンドモデル}
  観測方程式
  \begin{align*}
    y_{t} &= \theta_{1,t} + w_{t} \\
    \rightarrow \\
    y_{t} &= \left(\begin{array}{cc}1 & 0\end{array}\right)
      \left(\begin{array}{c}\theta_{1,t} \\ \theta_{2,t} \end{array}\right) + w_{t} 
  \end{align*}

  状態方程式
  \begin{align*}
    \theta_{1,t} &= \theta_{1,t-1} + \theta_{2,t-1} + v_{1,t} \\
    \theta_{2,t} &= \theta_{2,t-1} + v_{2,t} \\
    \rightarrow \\
    \left(\begin{array}{c}\theta_{1,t} \\ \theta_{2,t}\end{array}\right) &=
    \left(\begin{array}{cc}1 & 1 \\ 0 & 1 \end{array}\right)
    \left(\begin{array}{c}\theta_{1,t-1} \\ \theta_{2,t-1}\end{array}\right) +
    \left(\begin{array}{c}v_{1,t} \\ v_{2,t}\end{array}\right)
  \end{align*}

\end{frame}

%% Stan
\begin{frame}{Stan}

\end{frame}

%% gaussian_dlm_obs()
\begin{frame}{gaussian\_dlm\_obs}

  \textit{y}~\textasciitilde~\textbf{gaussian\_dlm\_obs}(\textit{F, G, V, W, m0, C0});

カルマンフィルタのパラメータ(分散)を推定する。

\end{frame}

\end{document}
